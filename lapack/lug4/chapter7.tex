\chapter{Troubleshooting}\label{troubleshooting}\index{troubleshooting}

\index{test suites}\index{reliability, see test suites}

\section{Locating Source of Trouble}

Successful use of LAPACK relies heavily on proper installation
and testing of LAPACK and the BLAS.  Frequently Asked Questions (FAQ) lists
are maintained in the {\em blas} and {\em lapack} directories on {\em netlib}
to answer some of the most common user questions.  For the user's convenience,
prebuilt LAPACK libraries are provided for a variety of computer
architectures in the following URL:
\begin{quote}
{\tt http://www.netlib.org/lapack/archives/}
\end{quote}
 
Test suites are provided for the BLAS and LAPACK.
It is highly recommended that each of these respective test
suites be run prior to the use of the library.
A comprehensive installation guide\index{documentation!installation guide} is
provided for LAPACK.  Refer to the {\em lapack} directory on {\em netlib} for
further information.

We begin this chapter by discussing a set of first-step debugging
hints to pinpoint where the problem is occurring.
Following these debugging hints, we discuss the types of
error messages that can be encountered during the execution of an LAPACK
routine.
 
If these suggestions do not help evaluate specific difficulties,
we suggest that the user review the following ``bug report checklist'' and then
feel free to contact the authors at {\tt lapack@cs.utk.edu}\index{bug
reports!mailing alias}\index{reporting bugs, see bug reports}.
The user should tell us the type of machine
on which the tests were run, the compiler and compiler options that were
used, and details of the BLAS library; also, the user should send a copy of
the input file, if appropriate.

\section{Bug Report Checklist}
\label{checklist}\index{bug reports!checklist}
 
When the user sends e-mail to our mailing alias, some of the first questions
we will ask are the following:
\begin{enumerate}
\item Have you run the BLAS and LAPACK test suites?
\item Have you checked the errata list ({\tt release\_notes}) on {\em netlib}?
\begin{quote}
{\tt http://www.netlib.org/lapack/release\_notes}
\end{quote}
 
\item If you are using an optimized BLAS library, have you
tried using the reference implementation from {\em netlib}?
 \begin{quote}
 {\tt http://www.netlib.org/blas/}
 \end{quote}
\item Have you attempted to replicate this error using the appropriate
LAPACK test code?
\end{enumerate}
 
\section{Installation Debugging Hints}
\index{debugging hints!installation}
 
If the user encounters difficulty in the installation process, we
suggest the following:
\begin{itemize}
 \item Obtain prebuilt LAPACK library on {\em netlib}
       for a variety of architectures.
       \begin{quote}
       {\tt http://www.netlib.org/lapack/archives/}
       \end{quote}
 \item Obtain sample {\tt make.inc} files for a variety of architectures
       in the {\tt LAPACK/INSTALL} directory in the lapack
       distribution tar file.
 \item Consult the LAPACK FAQ list on {\em netlib}.
 \item Consult the {\tt release\_notes} file in the {\em lapack} directory
       on {\em netlib}.  This file contains a list of known difficulties
       that have been diagnosed and corrected (or will be
       corrected in the next release), or reported to the vendor as in the case
       of optimized BLAS\index{debugging!release\_notes}.
 \item Always run the BLAS test suite to ensure that
       this library has been properly installed.  If a problem
       is detected in the BLAS library, try linking to the reference
       implementation in the {\em blas} directory on {\em netlib}.
\end{itemize}
 
\section{Common Errors in Calling LAPACK Routines}\label{seccommonerrors}

As a first step, the user should always carefully read the leading
comments of the LAPACK routine.  The leading comments give a detailed
description of all input/output arguments and can be viewed in the source
code, on the LAPACK webpage, or in this users' guide.

For the benefit of less experienced programmers, we list the most
common programming errors in calling an LAPACK routine.
These errors may cause the LAPACK routine to report a failure,
as described in Section~\ref{secfailures}\index{failures!common causes};
they may cause an error to be reported by the system;
or they may lead to wrong results --- see also
Section~\ref{secwrongresults}.

\begin{itemize}
\item wrong number of arguments
\item arguments in the wrong order
\item an argument of the wrong type (especially real and complex 
arguments of the wrong precision)
\item wrong dimensions for an array argument
\item insufficient space in a workspace argument
\item failure to assign a value to an input argument
\end{itemize}

Some modern compilation systems, as well as
software tools such as the portability checker in 
Toolpack~\cite{Toolpack}, can check that arguments agree in number and type;
and many compilation systems offer run-time detection
of errors such as an array element out-of-bounds or use of an
unassigned variable.

\section{Failures Detected by LAPACK Routines}\label{secfailures}
\index{failures}

There are two ways in which an LAPACK routine may report a failure to
complete a computation successfully.

\subsection{Invalid Arguments and XERBLA}
\index{error handler, XERBLA}\indexR{XERBLA}
If an illegal value is supplied for one of the input arguments to
an LAPACK routine, it will call the error handler XERBLA to write
a message to the standard output unit of the form:
\begin{verbatim}
 ** On entry to SGESV  parameter number  4 had an illegal value
\end{verbatim}
This particular message would be caused by passing to SGESV\indexR{SGESV} a value of LDA 
which was less than the value of the argument N.
The documentation for SGESV states the set of acceptable input values:
``LDA $\geq$ max(1,N).'' This is required in order that the
\index{arguments!LDA}
array A with leading dimension LDA can store an $n$-by-$n$ 
matrix.
The requirement is stated ``LDA $\geq$ max(1,N)'' 
rather than simply
``LDA $\geq$ N'' because LDA must always be at least 1, even if N = 0, 
to satisfy the 
requirements of standard Fortran; on some systems,
a zero or negative value of LDA would cause a run-time fault.
The arguments are checked in order, beginning with the first.
In the above example, it may --- from the user's point of view --- be the
value of N which is in fact wrong. 
Invalid arguments are often caused by the kind of error listed in 
Section~\ref{seccommonerrors}.

In the model implementation of XERBLA\indexR{XERBLA} which is supplied with LAPACK,
execution stops after the
message; but the call to XERBLA is followed by a RETURN statement
in the LAPACK routine, so that if the installer removes the
STOP statement in XERBLA, the result will be an immediate exit from the
LAPACK routine with a negative value of INFO.
It is good practice always to check for a nonzero value of INFO
on return from an LAPACK routine.
\index{arguments!INFO}
(We recommend however that XERBLA should not be modified to return control
to the calling routine, unless absolutely
necessary, since this would remove one of the built-in safety-features
of LAPACK.)

\subsection{Computational Failures and INFO $>$ 0}
\index{arguments!INFO}\index{INFO}
A positive value of INFO on return from an LAPACK routine indicates a
failure in the course of the algorithm. Common causes are:
\begin{itemize}
\item a matrix is singular (to working precision);
\item a symmetric matrix is not positive definite;
\item an iterative algorithm for computing eigenvalues or eigenvectors
fails to converge in the permitted number of iterations.
\end{itemize}
For example, if SGESVX\indexR{SGESVX} is called to solve a system of equations 
with a coefficient matrix that is approximately singular,
it may detect exact singularity at the $i^{th}$ stage of the $LU$
factorization, in which case it returns INFO = $i$;
or (more probably) it may compute an estimate of the reciprocal condition number
that is less than machine precision, in which case it returns INFO = $n$+1.
Again, the documentation should be consulted for a 
description of the error.  

When a failure with INFO $>$ 0 occurs, control is {\em always} returned
to the calling program; XERBLA is {\em not} called, and no error message
is written. 
It is worth repeating that it is good practice always to check for
a nonzero value of INFO on return from an LAPACK routine.

A failure with INFO $>$ 0 may indicate any of the following:

\begin{itemize}

\item an inappropriate routine was used: 
for example, if a routine fails because a symmetric matrix turns out not to be 
positive definite, consider using a routine for symmetric indefinite matrices.

\item a single precision routine was used when double precision was needed:
for example, if SGESVX\indexR{SGESVX} reports approximate singularity
(as illustrated above), the corresponding double precision routine DGESVX
may be able to solve the problem (but nevertheless the problem is
ill-conditioned).

\item a programming error occurred in generating the data supplied
to a routine: for example, even though theoretically a matrix should be
well-conditioned and positive definite, a programming error in generating
the matrix could easily destroy either of those properties.

\item a programming error occurred in calling the routine, of the kind 
listed in Section~\ref{seccommonerrors}.

\end{itemize}

\section{Wrong Results}\label{secwrongresults}\index{wrong results}

Wrong results from LAPACK routines are most often caused by incorrect usage.

It is also possible that wrong results are caused by a bug
outside of LAPACK, in the compiler or in one of the library routines,
such as the BLAS, that are linked with LAPACK.
Test procedures are available for both LAPACK and the BLAS, and
the LAPACK installation guide \cite{lawn41} should be consulted
for descriptions of the tests and for advice on resolving problems.

A list of known problems, compiler errors, and bugs in LAPACK routines is
maintained on {\em netlib}; see Chapter~\ref{chapessentials}.

Users who suspect they have found a new bug in an LAPACK routine are
encouraged to report it promptly to the developers as directed in
Chapter~\ref{chapessentials}.
The bug report should include a test case, a description of
the problem and expected results, and the actions, if any,
that the user has already taken to fix the bug.

\section{Poor Performance}

LAPACK relies on an efficient implementation of the BLAS\index{BLAS}.
We have tried to make
the performance of LAPACK ``transportable'' by performing most of
the computation within the Level 1, 2, and 3 BLAS, and by isolating
all of the machine-dependent tuning parameters
in a single integer function ILAENV\indexR{ILAENV}.

To avoid poor performance\index{avoiding poor performance} from LAPACK 
routines, note the
following recommendations\index{performance!recommendations}:

\begin{description}
\item[BLAS:]
One should use machine-specific optimized BLAS if they are available.
Many manufacturers and research institutions have developed, or are
developing, efficient versions of the BLAS for particular machines.
The BLAS enable LAPACK routines to achieve high performance
with transportable software.  Users are urged to determine whether such an
implementation of the BLAS exists for their platform. When
such an optimized implementation of the BLAS is available, it
should be used to ensure
optimal performance.
If such a
machine-specific implementation of the BLAS does not exist for a particular
platform, one should consider installing a publicly available
set of BLAS that requires only an efficient implementation of the
matrix-matrix multiply BLAS routine xGEMM.  Examples of such
implementations are \cite{dayde94a,kagstrom95b}.  A machine-specific and
efficient implementation of the routine GEMM can be automatically
generated by publicly available software such as \cite{atlas_sc98} and
\cite{lawn111}.
Although a reference implementation of the Fortran~77 BLAS is available
from the {\em blas} directory on {\em netlib}, these routines are not
expected to
perform as well as a specially tuned implementation on most high-performance
computers -- on some machines it may give much worse performance
-- but it allows users to run LAPACK software on machines
that do not offer any other implementation of the BLAS.

\item[ILAENV:]\indexR{ILAENV} For best performance, the LAPACK routine ILAENV
should be set with optimal tuning parameters for the machine being used.
The version of ILAENV provided with LAPACK supplies default values
for these parameters that give good, but not optimal, average
case performance on a range of existing machines.
In particular, the performance of xHSEQR is particularly sensitive to
\indexR{SHSEQR}\indexR{CHSEQR}
\index{performance!sensitivity}
the correct choice of block parameters; the same applies to the driver
routines which call xHSEQR, namely xGEES, xGEESX, xGEEV and xGEEVX.
\indexR{SGEES}\indexR{CGEES}
\indexR{SGEESX}\indexR{CGEESX}
\indexR{SGEEV}\indexR{CGEEV}
\indexR{SGEEVX}\indexR{CGEEVX}
Further details on setting parameters in ILAENV are found in
Section~\ref{secilaenv}.

\item[LWORK $\geq$ WORK(1):]
The performance of some routines depends on the amount of workspace
supplied. In such cases,
an argument, usually called WORK, is
provided, accompanied by an integer argument LWORK specifying its
length as a linear array. 
On exit, WORK(1) returns the amount of workspace required to use
the optimal tuning parameters.
If LWORK $<$ WORK(1), then insufficient workspace was provided
to use the optimal parameters, and the performance may be less
than possible.
One should check LWORK $\geq$ WORK(1) on return from
an LAPACK routine requiring user-supplied workspace to see if
enough workspace has been provided.
\index{performance!LWORK}
Note that the computation is performed correctly, even if the amount of
workspace is less than optimal, unless LWORK is reported as an
invalid value by a call to XERBLA as described in Section~\ref{secfailures}.

\item[xLAMCH:]\indexR{SLAMCH} Users should beware of the high cost of the {\em first}
call to the LAPACK auxiliary routine xLAMCH,
\index{installation!xLAMCH!cost of}
which computes
machine characteristics such as epsilon and the
smallest invertible number.
The first call dynamically determines a set of parameters defining
the machine's arithmetic, but these values are saved and subsequent
calls incur only a trivial cost.
For performance testing, the initial cost can be hidden by 
including a call to xLAMCH in the main program, before any calls to
LAPACK routines that will be timed.  A sample use of SLAMCH\indexR{SLAMCH} is
\begin{verbatim}
      XXXXXX = SLAMCH( 'P' )
\end{verbatim}
A cleaner but less portable solution is for the installer to
save the values computed by xLAMCH for a specific machine
and create a new version of xLAMCH with these constants set in
DATA statements, taking care that no accuracy is lost in the
translation.
\end{description}

