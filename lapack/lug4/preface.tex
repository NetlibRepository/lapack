\chapter*{Preface to the Fourth Edition}
\markboth{Preface to the Fourth Edition}{Preface to the Fourth Edition}
\addcontentsline{toc}{chapter}{Preface to the Fourth Edition}

Release 4.0 of LAPACK presents a number of new routines, and makes some changes
in the way this Users' Guide is presented.

The following new routines have been introduced at intermediate updates to the software
(3.1, 3.2, 3.3, 3.4, 3.5):

\begin{enumerate}

\item routines for computing bounds on the error in the computed solution
to linear equations in the -GE-, -GB-, -SY-, -HE- and -PO- formats,
and refining the solution,
which is computed in twice the standard precision, resulting in errors of  only $O(\epsilon)$
(xyySVXX, xyyEQUB, xyyRFSX);
the routines call auxiliaries from the XBLAS package to achieve this

\item mixed precision routines (DSGESV, ZCGESV, DSPOSV, ZCPOSV) 
which require the problem to be presented and the solution to be returned in double precision,
but perform much of the computation internally in single precision

\item routines for the solution of symmetric or Hermitian linear equations,
using the bounded Bunch-Kaufman rook-pivoting method
which ensures greater stability
(\mbox{xSYSV\_ROOK}, \linebreak
 \mbox{xHESV\_ROOK}, 
 \mbox{xSYTRF\_ROOK}, 
 \mbox{xHETRF\_ROOK},
 \mbox{xSYCON\_ROOK}, 
 \mbox{xHECON\_ROOK}, \linebreak
 \mbox{xSYTRS\_ROOK}, 
 \mbox{xHETRS\_ROOK},
 \mbox{xSYTRI\_ROOK}, 
 \mbox{xHETRI\_ROOK})

\item routines which call Level 3 BLAS to solve symmetric or Hermitian linear equations
 or invert the matrix, thereby markedly increasing their speed
(xSYTRS2, xHETRS2, xSYTRI2, xHETRI2)

\item a routine which computes the Cholesky factorization
of a positive semi-definite matrix with complete pivoting
(xPSTRF)

\item routines for solving symmetric or Hermitian positive definite linear equations,
where the matrix is stored in Rectangular Full Packed format
in only half as much storage, but still using Level 3 BLAS
(xPFTRF, xPFTRS, xPFTRI, xTFTRI)
and routines for converting to and from Rectangular Full Packed format
(xTFTTP, xTFTTR, xTPTTF, xTPTTR, xTRTTF, xTRTTP)

\item a routine which computes a Householder $QR$ transformation 
with non-negative diagonal of $R$ (xGEQRFP)

\item routines using the compact WY representation of Householder matrices
(xGEQRT, xGEQRT3, xGEMQRT, xTPQRT, xTPMQRT)

\item routines to compute the singular values and vectors of a real matrix
by a one-sided Jacobi method, which in some cases, but not all, is faster than other routines
(SGEJSV, SGESVJ)

\item a new routine which computes some or all of the eigenvalues, and optionally the eigenvectors,
of a symmetric tridiagonal matrix, using the relatively robust representation (xSTEMR)

\item routines to compute the complete CS decomposition of an orthogonal matrix
or the 2-by-1 CS decomposition of a matrix with orthonormal columns
(\mbox{xORCSD/xUNCSD}, \linebreak
 \mbox{xORCSD2BY1/xUNCSD2BY1}, 
 \mbox{xORBDBn/xUNBDBn} where $n$ is blank, 1, 2, 3 or 4, \linebreak
 \mbox{xBBCSD})

\end{enumerate}

The following major additions/modifications to this Users' Guide have been made.

Chapter 1 (Essentials) now states that LAPACK subroutines can be called from programs written
in either Fortran 90 or C. For programs written in C, an extra header file {\tt lapack.h} must be 
included. Full details and examples are given in a new Chapter~\ref{chapcinterface}.

Chapter 2 (Contents of LAPACK) has been expanded to discuss the new routines.

Chapter 3 (Performance of LAPACK) has been shortened in that it no longer presents
timings from specific hardware. The timing programs are no longer made available for
distribution.

Chapter 4 (Accuracy and Stability) has been extended to include a discussion of the new
routines.

Chapter 5 (Documentation and Software Conventions) now includes a section on
Rectangular Full Packed format. 

Chapter 6 (Installing LAPACK Routines) has been shortened,
like Chapter~\ref{chapperformance}, by excluding
timings from specific hardware.

Chaper 8 (The LAPACKE C Interface to LAPACK) is a new chapter which describes how LAPACK
routines may be called from programs written in C with the aid of a header file {\tt lapacke.h},
and presents some examples.
 
Appendix A (Index of Driver and Computational Routines) has been divided in two:
\ref{secindexdrivers} an Index of Driver Routines;
and \ref{secindexcomp} an Index of Computational Routines.

Appendix B (Index of Auxiliary Routines) has been divided in four:
\ref{secindexauxilunblocked} an Index of Unblocked Routines (the blocked routines are listed in Appendix~\ref{secindexdrivers} or \ref{secindexcomp});
\ref{secindexauxilmisc} an Index of Miscellaneous Auxiliary Routines; this is the great majority,
most (but not quite all)  having names with the 2nd  and 3rd characters -LA-:
\ref{secindexauxilutility} an Index of Utility Routines, not dependent on precision;
and \ref{secindexauxilblasext} an Index of Extensions to the BLAS.

Appendix D (Converting from LINPACK or EISPACK) has been deleted.

Appendix E (LAPACK Working Notes) has also been deleted;
all Working Notes that are relevant to the current edition of LAPACK
are mentioned in the bibliography.

Throughout this Users' Guide, any mention of a single precision routine with name beginning S-
must be understood to apply also to the double precision routine with name beginning D-;
any mention of a single precision complex routine with name beginning C-
must be understood to apply also to the double precision complex routine with name beginning Z-.

There is just one exception to this rule: the routines DSGESV and DSPOSV and
their complex analogues ZCGESV and ZCPOSV are {\bf mixed precision} routines:
they require the problem to be presented in double precision and the solution to be returned in
double precision, but within the routines computation is done in single precision with
iterative refinement (unless that fails, in which case a double precision solution is computed).

We would like to thank the following people, who have made signification contributions to
the software and to this edition of the Users' Guide:

Michael Baudin,
Karen Braman,
Zvonimir Bujanovi\'{c},
Alfredo Buttari,
Ralph Byers,
Bobby Cheng,
Zlatko Drma\v{c},
Fred Gustavson,
Deaglan Halligan,
Yozo Hida,
Rodney James,
Igor Kozachenko,
Daniel Kressner,
Jakub Kurzak,
Julie Langou,
Julien Langou,
Bradley Lowery,
Craig Lucas,
Piotr Luszczek,
Osni Marques,
Beresford Parlett,
Jason Riedy,
Robert L. Smith,
Brian Sutton,
Kresimir Veseli\'{c},
Christof V\"{o}mel,
Jerzy Wasniewski.
  
\chapter*{Preface to the Third Edition}
\markboth{Preface to the Third Edition}{Preface to the Third Edition}
\addcontentsline{toc}{chapter}{Preface to the Third Edition}

Since the release of version 2.0 of the LAPACK software and the second edition
of the Users' Guide in 1994, LAPACK has been expanded further and has become an
even wider community effort.
The publication of this third edition of the
Users' Guide coincides with the release of version 3.0
of the LAPACK software.  Some of the software contributors to this release
were not original LAPACK authors, and thus their names have been
credited in the routines to which they contributed.

Version 3.0 of LAPACK introduces new routines, as well as extending
the functionality of existing routines.  The most significant
new routines and functions are:
\begin{enumerate}
 \item a faster singular value decomposition (SVD),
    computed by divide and conquer (xGESDD)
 \item faster routines for solving rank-deficient least squares problems:
    \begin{itemize}
     \item  using QR with column pivoting (xGELSY, based on xGEQP3)
     \item  using the SVD, based on divide and conquer (xGELSD)
    \end{itemize}
 \item new routines for the generalized symmetric eigenproblem:
     \begin{itemize}
      \item faster routines based on divide and conquer
         (xHEGVD/xSYGVD, xHPGVD/xSPGVD, xHBGVD/xSBGVD) 
      \item routines based on
         bisection/inverse iteration to compute more efficiently
         a subset of the spectrum
        (xHEGVX/xSYGVX, xHPGVX/xSPGVX, xHBGVX/xSBGVX)
      \end{itemize}
\item  faster routines for the symmetric eigenproblem using the
     ``relative robust representation'' algorithm
     (xSYEVR/xHEEVR, SSTEVR, xSTEGR)
 \item new simple and expert drivers for the generalized nonsymmetric
    eigenproblem, including error bounds (xGGES, xGGEV, xGGESX, xGGEVX)
 \item a solver for the generalized Sylvester equation (xTGSYL)
 \item computational routines (xTGEXC, xTGSEN, xTGSNA)
 \item a blocked version of xTZRQF and associated routines (xTZRZF, xORMRZ/xUNMRZ)
\end{enumerate}

One of the primary design features of the LAPACK library is that all
releases are backward compatible.  A user's program calling LAPACK will
never fail because of a new release of the library.  As a result,
however, the calling sequences (or amount of workspace required) to existing
routines cannot be altered.  Therefore, if a performance enhancement requires a
modification of this type, a new routine must be created.  There are
several routines included in LAPACK, version 3.0, that fall into this
category.  Specifically,
\begin{itemize}
\item xGEGS is deprecated and replaced by routine xGGES
\item xGEGV is deprecated and replaced by routine xGGEV
\item xGELSX is deprecated and replaced by routine xGELSY
\item xGEQPF is deprecated and replaced by routine xGEQP3
\item xTZRQF is deprecated and replaced by routine xTZRZF
\item xLATZM is deprecated and replaced by routines xORMRZ/xUNMRZ
\end{itemize}
The ``old'' version of the routine is still included in the
library but the user is advised to upgrade to the ``new'' faster
version.  References to the ``old'' versions are removed from this
users' guide.

In addition to replacing the above list of routines, there are a number of
other significantly faster new driver routines that we recommend in place of
their older counterparts listed below. We continue to include the older drivers
in this users' guide because the old drivers may use less workspace than the
new drivers, and because the old drivers may be faster in certain special cases
(we will continue to improve the new drivers in a future release until they
completely replace their older counterparts):
\begin{itemize}
\item xSYEV/xHEEV and xSYEVD/xHEEVD should be replaced by xSYEVR/xHEEVR
\item xSTEV and xSTEVD should be replaced by xSTEVR
\item xSPEV/xHPEV should be replaced by xSPEVD/xHPEVD
\item xSBEV/xHBEV should be replaced by xSBEVD/xHBEVD
\item xGESVD should be replaced by xGESDD
\item xSYGV/xHEGV should be replaced by xSYGVD/xHEGVD
\item xSPGV/xHPGV should be replaced by xSPGVD/xHPGVD
\item xSBGV/xHBGV should be replaced by xSBGVD/xHBGVD
\end{itemize}

This release of LAPACK introduces routines that exploit IEEE arithmetic.
We have a prototype running of a new algorithm (xSTEGR), which may be the
ultimate solution for the symmetric eigenproblem on both parallel and serial
machines.
This algorithm has been incorporated into the drivers xSYEVR, xHEEVR and xSTEVR
for the symmetric eigenproblem, and will be propagated into the generalized
symmetric definite eigenvalue problems, the SVD, the generalized SVD and the
SVD-based least squares solver.
Refer to section~\ref{subseccompsep} for further information.
We expect to also propagate this algorithm into ScaLAPACK.

We have also incorporated the {\tt LWORK=-1} query capability into this
release of LAPACK, whereby a user can request the amount of workspace required
for a routine.  For complete details, refer to section~\ref{lworkquery}.

All LAPACK routines reflect the current version number with the
date on the routine indicating when it was last modified.
For more information on revisions to the LAPACK software or this Users'
Guide please refer to the LAPACK {\tt release\_notes} file on netlib.
Instructions for obtaining this file can be found in
Chapter~\ref{chapessentials}.

The following additions/modifications have been made to this third edition
of the Users' Guide:

Chapter~\ref{chapessentials} (Essentials) 
includes updated information on accessing LAPACK and related projects 
via the World Wide Web.

Chapter~\ref{chapcontents} (Contents of LAPACK) has been expanded to discuss
the new routines.

Chapter~\ref{chapperformance} (Performance of LAPACK) has been updated
with performance results for version 3.0 of LAPACK.

Chapter~\ref{chapaccstab} (Accuracy and Stability) has been extended
to include error bounds for generalized least squares.

Appendices~\ref{chapindexuser} and~\ref{chapindexauxil} have
been expanded to cover the new routines.

Appendix~E (LAPACK Working Notes) lists a number of new
Working Notes, written during the LAPACK~2 and ScaLAPACK 
projects (see below) and published by the University of Tennessee.
The Bibliography has been updated to give
the most recent published references.

The Specifications of Routines have been extended and updated to cover
the new routines and revisions to existing routines.

The original LAPACK project was funded by the NSF. Since its completion,
four follow-up projects, LAPACK~2, ScaLAPACK, ScaLAPACK~2 and LAPACK~3 have
been funded in the U.S. by the NSF and ARPA in 1990--1994, 1991--1995,
1995--1998, and 1998--2001, respectively.  

In addition to making possible
the additions and extensions in this release, these grants have 
supported the following closely related activities.

A major effort is underway
to implement LAPACK-type algorithms for distributed memory
\index{distributed memory} machines.
As a result of these efforts,
several new software items are now available on netlib.  The new
items that have been introduced are distributed memory versions of the
core routines from LAPACK; sparse Gaussian elimination -- SuperLU, SuperLU\_MT,
and distributed-memory SuperLU; a fully parallel package to solve a symmetric
positive definite sparse linear system on a message passing
multiprocessor using Cholesky factorization\index{CAPSS}; a package based on
Arnoldi's method for solving large-scale nonsymmetric, symmetric, and
generalized algebraic eigenvalue problems\index{ARPACK}; 
and templates for sparse
iterative methods for solving $Ax=b$\index{ParPre}.
For more information on the
availability of each of these packages, consult the following URLs:
\begin{quote}
{\tt http://www.netlib.org/scalapack/} \\
{\tt http://www.netlib.org/linalg/}
\end{quote}

Alternative language interfaces to LAPACK (or translations/conversions
of LAPACK) are available in Fortran~95, C, and Java. 
For more information consult Section~\ref{relsoftware} or the following URLs:
\begin{quote}
{\tt http://www.netlib.org/lapack90/} \\
{\tt http://www.netlib.org/clapack/} \\
{\tt http://www.netlib.org/java/f2j/}
\end{quote}

The performance results presented in this book were obtained using
computer resources at various sites:
\begin{itemize}

\item
Compaq AlphaServer DS-20, donated by Compaq Corporation, and
located at the Innovative Computing Laboratory, in
the Department of Computer Science, University of Tennessee, Knoxville.

\item
IBM Power~3, donated by IBM, and located at the Innovative Computing
Laboratory, in the Department of Computer Science, University of
Tennessee, Knoxville.

\item
Intel Pentium III, donated by Intel Corporation, and
located at the Innovative Computing Laboratory, in
the Department of Computer Science, University of Tennessee, Knoxville.

\item
Clusters of Pentium IIs, PowerPCs, and Alpha EV56s, located at the LIP
(Laboratoire de l'Informatique du Parall\'{e}lisme), ENS (\'{E}cole
Normale Sup\'{e}rieure), Lyon, France.

\item
SGI Origin 2000, located at the Army Research Laboratory
in Aberdeen Proving Ground, Maryland, and supported by the DoD High
Performance Computing Modernization Program ARL Major Shared Resource Center
through Programming Environment and Training (PET) under Contract Number
DAHC-94-96-C-0010, Raytheon E-Systems, subcontract no. AA23.

\end{itemize}


We would like to thank the following people, who were either not
acknowledged in previous editions, or who have made significant
additional contributions to this edition:
 
Henri Casanova,
Tzu-Yi Chen,
David Day,
Inderjit Dhillon,
Mark Fahey,
Patrick Geoffray,
Ming Gu,
Greg Henry,
Nick Higham,
Bo K{\aa}gstr\"{o}m,
Linda Kaufman,
John Lewis,
Ren-Cang Li,
Osni Marques,
Rolf Neubert,
Beresford Parlett,
Antoine Petitet,
Peter Poromaa,
Gregorio Quintana,
Huan Ren,
Jeff Rutter,
Keith Seymour,
Vasile Sima,
Ken Stanley,
Xiaobai Sun,
Fran\c{c}oise Tisseur,
Zachary Walker, and
Clint Whaley. \\

\vspace{1in}

\chapter*{Preface to the Second Edition}
\markboth{Preface to the Second Edition}{Preface to the Second Edition}
\addcontentsline{toc}{chapter}{Preface to the Second Edition}

Since its initial public release in February 1992, LAPACK has expanded
in both depth and breadth.  LAPACK is now available in both Fortran and C.
The publication of this second edition of the
Users' Guide coincides with the release of version 2.0
of the LAPACK software.

This release of LAPACK introduces
new routines and extends the functionality of 
existing routines.  Prominent among the new routines are 
driver and computational routines for 
the generalized nonsymmetric eigenproblem, 
generalized linear least squares problems, 
the generalized singular value decomposition, 
a generalized banded symmetric definite eigenproblem,
and divide and conquer methods for symmetric eigenproblems.  
Additional computational routines include 
the generalized QR and RQ factorizations and
reduction of a band matrix to bidiagonal form.

Added functionality has been incorporated into the expert driver
routines that involve equilibration (xGESVX, xGBSVX, xPOSVX, xPPSVX,
and xPBSVX).  The option FACT = 'F' now permits the user to input a
prefactored, pre-equilibrated matrix.  The expert drivers xGESVX and xGBSVX
now return the reciprocal of the pivot growth from Gaussian
elimination.  xBDSQR has been modified to compute singular values of
bidiagonal matrices much more quickly than before, provided singular
vectors are not also wanted.
The least squares driver
routines xGELS, xGELSS, and xGELSX now make available the residual 
root-sum-squares for each right hand side.  

All LAPACK routines reflect the current version number with the
date on the routine indicating when it was last modified.
For more information on revisions to the LAPACK software or this Users'
Guide please refer to the LAPACK {\tt release\_notes} file on netlib.
Instructions for obtaining this file can be found in
Chapter~\ref{chapessentials}.

On-line manpages (troff files) for LAPACK routines, as well as for most of 
the BLAS routines, are available on netlib.  Refer to
Section~\ref{documentation} for further details.

We hope that future releases of LAPACK will include routines for reordering
eigenvalues in the generalized Schur factorization; 
solving the generalized Sylvester
equation; computing condition numbers for the generalized eigenproblem
(for eigenvalues, eigenvectors, clusters of eigenvalues, and deflating
subspaces);
fast algorithms for the singular value decomposition based on 
divide and conquer;
high accuracy methods for symmetric eigenproblems and the SVD based
on Jacobi's algorithm;
updating and/or downdating for linear least squares problems; 
computing singular values by bidiagonal
bisection; and computing singular vectors by bidiagonal inverse iteration.

The following additions/modifications have been made to this second edition
of the Users' Guide:

Chapter~\ref{chapessentials} (Essentials) 
now includes information on accessing LAPACK via the World
Wide Web.

Chapter~\ref{chapcontents} (Contents of LAPACK) has been expanded to discuss
new routines.

Chapter~\ref{chapperformance} (Performance of LAPACK) has been updated
with performance results from version 2.0 of LAPACK.  In addition,
a new section entitled ``LAPACK Benchmark'' has been introduced to
present timings for several driver routines.

Chapter~\ref{chapaccstab} (Accuracy and Stability) has been simplified and
rewritten.
Much of the theory and other details have been separated into
``Further Details'' sections.  Example Fortran code segments are
included to demonstrate the calculation of error bounds using LAPACK.

Appendices~\ref{chapindexuser},~\ref{chapindexauxil} and D have
been expanded to cover the new routines.

Appendix~E (LAPACK Working Notes) lists a number of new
Working Notes, written during the LAPACK~2 and ScaLAPACK 
projects (see below) and published by the University of Tennessee.
The Bibliography has been updated to give
the most recent published references.

The Specifications of Routines have been extended and updated to cover
the new routines and revisions to existing routines.

The Bibliography and Index have been moved to the end of the book.
The Index has been expanded into two indexes:  Index by Keyword and
Index by Routine Name.  Occurrences of LAPACK, LINPACK, and EISPACK
routine names have been cited in the latter index.

The original LAPACK project was funded by the NSF. Since its completion,
two follow-up projects, LAPACK~2 and ScaLAPACK, have been funded in the
U.S. by the NSF and ARPA in 1990--1994 and 1991--1995, respectively.  
In addition to making possible
the additions and extensions in this release, these grants have 
supported the following closely related activities.

A major effort is underway
to implement LAPACK-type algorithms for distributed memory
\index{distributed memory} machines.
As a result of these efforts,
several new software items are now available on netlib.  The new
items that have been introduced are distributed memory versions of the
core routines from LAPACK; a fully parallel package to solve a symmetric
positive definite sparse linear system on a message passing
multiprocessor using Cholesky factorization\index{CAPSS}; a package based on
Arnoldi's method for solving large-scale nonsymmetric, symmetric, and
generalized algebraic eigenvalue problems\index{ARPACK}; 
and templates for sparse
iterative methods for solving $Ax=b$.
For more information on the
availability of each of these packages, consult the {\tt scalapack}
and {\tt linalg} indexes on netlib via {\tt netlib@www.netlib.org}.

We have also explored the advantages of IEEE floating point arithmetic
\cite{ieee754} 
in implementing linear algebra routines. The accurate rounding properties
and ``friendly'' exception handling capabilities of IEEE arithmetic
permit us to write faster, more robust versions of several algorithms in
LAPACK. Since all machines do not yet implement IEEE arithmetic, these
algorithms are not currently part of the library \cite{demmelli93},
although we expect them to be in the future.  For more information,
please refer to Section ~\ref{relsoftware}.

LAPACK has been translated from Fortran into C and, in addition,
a subset of the LAPACK routines has been implemented in C++\index{LAPACK++}.
For more information on obtaining the C or C++ versions of LAPACK, consult 
Section ~\ref{relsoftware} or the 
{\tt clapack} or {\tt c++} indexes on netlib via {\tt netlib@www.netlib.org}. 

We deeply appreciate the careful scrutiny of those individuals who
reported mistakes, typographical errors, or shortcomings in the first
edition.

We acknowledge with gratitude the support which we have received from the
following organizations and the help of individual members of their staff:
Cray Research Inc.;
NAG Ltd.

We would additionally like to thank the following people, who were not
acknowledged in the first edition, for their contributions:

Fran\c{c}oise Chatelin,
Inderjit Dhillon,
Stan Eisenstat,
Vince Fernando,
Ming Gu,
Rencang Li,
Xiaoye Li,
George Ostrouchov,
Antoine Petitet,
Chris Puscasiu,
Huan Ren,
Jeff Rutter,
Ken Stanley,
Steve Timson,
and Clint Whaley.

\vspace{1in}

\chapter*{Preface to the First Edition}
\markboth{Preface to the First Edition}{Preface to the First Edition}
\addcontentsline{toc}{chapter}{Preface to the First Edition}

The development of LAPACK was a natural step after specifications of
the Level 2 and 3 BLAS were drawn up in 1984--86 and 1987--88.
Research on block algorithms had been ongoing for several years,
but agreement on the BLAS made it possible to construct a new
software package to take the place of LINPACK and EISPACK,
which would achieve much greater efficiency on modern high-performance
computers.
This also seemed to be a good time to implement a number of
algorithmic advances that had been made since LINPACK and EISPACK
were written in the 1970's.
The proposal for LAPACK was submitted while the Level 3 BLAS were
still being developed and funding was obtained from the National 
Science Foundation (NSF) beginning in 1987.

LAPACK is more than just a more efficient update of its popular predecessors.
It extends the functionality of LINPACK and EISPACK by including:
driver routines for linear systems;
equilibration, iterative 
refinement and error 
bounds for linear systems; 
routines for computing and re-ordering the Schur 
factorization;
and condition estimation routines for eigenvalue problems.
LAPACK improves on the accuracy of the standard algorithms in EISPACK
by including high accuracy algorithms for finding singular values and
eigenvalues of bidiagonal and tridiagonal matrices, respectively,
that arise in SVD and symmetric eigenvalue problems.

We have tried to be consistent with our documentation and
coding style throughout LAPACK in the hope that LAPACK will serve
as a model for other software development efforts.
In particular, we hope that LAPACK and this guide will
be of value in the classroom.
But above all, LAPACK has been designed to be used for serious
computation, especially
as a source of building blocks for larger applications.

The LAPACK project has been a research project on achieving good
performance in a portable way over a large class of modern computers.
This goal has been achieved, subject to the following qualifications.
For optimal performance, it is necessary, first, that 
the BLAS are implemented efficiently on the
target machine, and second, that a small number of tuning parameters
(such as the block size) have been set to suitable values
(reasonable default values are provided). 
Most of the LAPACK code is written in standard Fortran 77,
but the double precision complex data type is not part of the standard,
so we have had to make some assumptions about the names of intrinsic
functions that do not hold on all machines (see section~\ref{chapinstallsec1}).
Finally, our rigorous testing suite included test problems scaled
at the extremes of the arithmetic range,
which can vary greatly from machine to machine.
On some machines, we have had to restrict the range more than on others.

Since most of the performance improvements in LAPACK come from
restructuring the algorithms to use the Level 2 and 3 BLAS,
we benefited greatly by having access from the early stages of the
project to a complete set of BLAS developed for the CRAY machines
by Cray Research.  Later, the BLAS library developed by IBM for the
IBM RISC/6000 was very helpful in proving the worth of block algorithms
and LAPACK on ``super-scalar'' workstations.  Many of our test sites, both
computer vendors and research institutions, also worked on optimizing
the BLAS and thus helped to get good performance from LAPACK.
We are very pleased at the extent to which the user community has
embraced the BLAS, not only for performance reasons, but also because
we feel developing software around a core set of common routines
like the BLAS is good software engineering practice.

A number of technical reports were written during the development of
LAPACK and published as LAPACK Working Notes, initially by Argonne
National Laboratory and later by the University of Tennessee.
Many of these reports later appeared as journal articles. 
Appendix~E lists the LAPACK Working Notes,
and the Bibliography gives the
most recent published reference.

A follow-on project, LAPACK~2, has been funded in the U.S. by the NSF
and DARPA.
One of its aims will be to add a modest amount of additional
functionality to the current LAPACK package --- for example, 
routines for the generalized SVD and additional routines for 
generalized eigenproblems. These routines will be included in a future
release of LAPACK when they are available.
LAPACK~2 will also produce routines which implement LAPACK-type
algorithms for distributed memory machines, routines which take special 
advantage of IEEE arithmetic, and versions of parts of LAPACK in
C and Fortran 90.
The precise form of these other software packages which will
result from LAPACK 2 has not yet been decided.

As the successor to LINPACK and EISPACK, LAPACK has drawn
heavily on both the software and documentation from those collections.
The test and timing software for the Level 2 and 3 BLAS was
used as a model for the LAPACK test and timing software, and in
fact the LAPACK timing software includes the BLAS timing software
as a subset.
Formatting of the software and conversion from single to double
precision was done using Toolpack/1~\cite{Toolpack}, which was
indispensable to the project.
We owe a great debt to our colleagues
who have helped create the infrastructure of scientific computing on
which LAPACK has been built.

The development of LAPACK was primarily supported by NSF grant 
ASC--8715728. 
Zhaojun Bai
had partial support from DARPA grant F49620--87--C0065;
Christian Bischof was supported by the Applied
Mathematical Sciences subprogram of the Office of Energy Research,
U.S. Department of Energy, under contract W--31--109--Eng--38; James
Demmel had partial support from NSF grant DCR--8552474;
and
Jack Dongarra had partial support from the
Applied Mathematical Sciences subprogram
of the Office of Energy Research, U.S.
Department of Energy, under Contract
DE--AC05--84OR21400.

The cover was designed by Alan Edelman at UC Berkeley who discovered the matrix
by performing Gaussian elimination on a certain 20-by-20 Hadamard matrix.

We acknowledge with gratitude the support which we have received from the
following organizations, and the help of individual members of their staff:
Cornell Theory Center;
Cray Research Inc.;
IBM ECSEC Rome;
IBM Scientific Center, Bergen;
NAG Ltd.

We also thank many, many people who have contributed code,
criticism, ideas and encouragement.
We wish especially to acknowledge the contributions of:
Mario Arioli,
Mir Assadullah,
Jesse Barlow,
Mel Ciment,
Percy Deift,
Augustin Dubrulle,
Iain Duff,
Alan Edelman,
Victor Eijkhout,
Sam Figueroa,
Pat Gaffney,
Nick Higham,
Liz Jessup,
Bo K\aa gstr\"{o}m,
Velvel Kahan,
Linda Kaufman,
L.-C. Li,
Bob Manchek,
Peter Mayes,
Cleve Moler,
Beresford Parlett,
Mick Pont,
Giuseppe Radicati,
Tom Rowan,
Pete Stewart,
Peter Tang,
Carlos Tomei,
Charlie Van Loan,
Kre\v{s}imir Veseli\'{c},
Phuong Vu,
and Reed Wade.

Finally we thank all the test sites who received three preliminary
distributions of LAPACK software and who ran an extensive series of
test programs and timing programs for us; their efforts have influenced
the final version of the package in numerous ways.

